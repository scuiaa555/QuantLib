The Quant\+Lib project (\href{http://quantlib.org}{\tt http\+://quantlib.\+org}) is aimed at providing a comprehensive software framework for quantitative finance. Quant\+Lib is a free/open-\/source library for modeling, trading, and risk management in real-\/life.

Quant\+Lib is Non-\/\+Copylefted Free Software and O\+SI Certified Open Source Software.

\subsection*{Download and usage }

Quant\+Lib can be downloaded from \href{http://quantlib.org/download.shtml}{\tt http\+://quantlib.\+org/download.\+shtml}; installation instructions are available at \href{http://quantlib.org/install.shtml}{\tt http\+://quantlib.\+org/install.\+shtml} for most platforms.

Documentation for the usage and the design of the Quant\+Lib library is available from \href{http://quantlib.org/docs.shtml}{\tt http\+://quantlib.\+org/docs.\+shtml}.

A list of changes for each past versions of the library can be browsed at \href{http://quantlib.org/reference/history.html}{\tt http\+://quantlib.\+org/reference/history.\+html}.

\subsection*{Questions and feedback }

Bugs can be reported as a Git\+Hub issue at \href{https://github.com/lballabio/QuantLib/issues}{\tt https\+://github.\+com/lballabio/\+Quant\+Lib/issues}; if you have a patch available, you can open a pull request instead (see \char`\"{}\+Contributing\char`\"{} below).

You can also use the {\ttfamily quantlib-\/users} and {\ttfamily quantlib-\/dev} mailing lists for feedback, questions, etc. More information and instructions for subscribing are at \href{http://quantlib.org/mailinglists.shtml}{\tt http\+://quantlib.\+org/mailinglists.\+shtml}.

\subsection*{Contributing }

The easiest way to contribute is through pull requests on Git\+Hub. Get a Git\+Hub account if you don\textquotesingle{}t have it already and clone the repository at \href{https://github.com/lballabio/QuantLib}{\tt https\+://github.\+com/lballabio/\+Quant\+Lib} with the \char`\"{}\+Fork\char`\"{} button in the top right corner of the page. Check out your clone to your machine, code away, push your changes to your clone and submit a pull request; instructions are available at \href{https://help.github.com/articles/fork-a-repo}{\tt https\+://help.\+github.\+com/articles/fork-\/a-\/repo}. (In case you need them, more detailed instructions for creating pull requests are at \href{https://help.github.com/articles/using-pull-requests}{\tt https\+://help.\+github.\+com/articles/using-\/pull-\/requests}, and a basic guide to Git\+Hub is at \href{https://guides.github.com/activities/hello-world/}{\tt https\+://guides.\+github.\+com/activities/hello-\/world/}.

It\textquotesingle{}s likely that we won\textquotesingle{}t merge your code right away, and we\textquotesingle{}ll ask for some changes instead. Don\textquotesingle{}t be discouraged! That\textquotesingle{}s normal; the library is complex, and thus it might take some time to become familiar with it and to use it in an idiomatic way.

We\textquotesingle{}re looking forward to your contributions. 